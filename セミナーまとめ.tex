 %"\"をコマンドの前に基本入れる
    %↓↓最初にプリアンブルと呼ばれる基本情報を入れる
    \documentclass[10.5pt]{jsarticle}
    \usepackage{amsmath}  %usepackageでパッケージをダウンロードする。"amsmath"は複雑な数式を扱うときに使う
    \usepackage{amssymb}  %amsmathとamssymbはとりあえず入れておく
    \usepackage{comment}  %複数行にわたるコメントを打ち込むときに必要
    \usepackage[dvipdfmx]{graphicx}
    \usepackage{mathtools}

    \usepackage{amsthm}  %定理環境 https://mathlandscape.com/latex-amsthm/
    \theoremstyle{plain} %書体等の形式
    \newtheorem{thm}{Theorem}[section] %"thm"と打てば定理が"Theorem"がでる[section]はTheorem1.2のようにセクション番号まで番号を振ることができる設定
    \newtheorem{dfn}[thm]{Definition} %真ん中の[thm]はthmと同じ通し番号を振るということ
    \newtheorem{lem}[thm]{Lemma}
    \newtheorem{rem}[thm]{Remark}
    \newtheorem*{rem*}{Remark} %通し番号を振らないときは*をつける
    \newtheorem{cor}[thm]{Corollary}
    \newtheorem{prop}[thm]{Proposition}
    \newtheorem{fact}[thm]{Fact}

    %定理番号は1.1から順番に自動で振られるが\setcounter{section}{4}などと入力することで途中からの番号をつけることができる


    \title{線形写像と行列の関係}
    \author{中井竜太郎\\\small{学籍番号 0530-37-0919}}     %これらはあとで\maketitleでタイトルに表示する

    %\pagestyle{headings}  %ページスタイルの設定(別に要らない)

    %-------------------------------------------------------------------------------------------------------
\begin{document}%ここから本文を書き始める。コマンド等を何も使わなければそのまま文字が出力される
    \begin{dfn}
        $f\in L^1(\Omega)$とする
       $$\int_{\Omega} |Df| \coloneqq \sup \left\{\int_{\Omega} f \mathrm{div} gdx \Big|g=(g_1,g_2,...,g_n)\in C_c^1(\Omega : \mathbb{R}^n ),|g(x)|\leqq1 ~ \text{for}~ x\in \Omega\right\}$$
       と定める$\int_{\Omega} |Df|< \infty $の時,$f$は有界変動であるという.\\
       $f\in L^1(\Omega)$で有界変動であるもの全体の集合を$BV(\Omega)$とかく.
    \end{dfn}
    \begin{prop}
    $f\in W_{loc}^{1,1}(\Omega)$のとき
    $$\int_{\Omega} |Df|=\int_{\Omega} |Df|dx$$
    ここで右辺の$Df$は弱微分の意味での$f$のgradientを指す.\\
    特に$f\in C^1(\Omega)$の時に上式が成り立つ.
    \end{prop}
    \begin{rem}
    a

    \end{rem}
    git add ファイル名してgit commit -m "コメント"してgit pushする





    \end{document}