 %"\"をコマンドの前に基本入れる
    %↓↓最初にプリアンブルと呼ばれる基本情報を入れる
    \documentclass[10.5pt]{jsarticle}
    \usepackage{amsmath}  %usepackageでパッケージをダウンロードする。"amsmath"は複雑な数式を扱うときに使う
    \usepackage{amssymb}  %amsmathとamssymbはとりあえず入れておく
    \usepackage{comment}  %複数行にわたるコメントを打ち込むときに必要
    \usepackage[dvipdfmx]{graphicx}
    \usepackage{mathtools}

    \usepackage{amsthm}  %定理環境 https://mathlandscape.com/latex-amsthm/
    \theoremstyle{plain} %書体等の形式
    \newtheorem{thm}{Theorem}[section] %"thm"と打てば定理が"Theorem"がでる[section]はTheorem1.2のようにセクション番号まで番号を振ることができる設定
    \newtheorem{dfn}[thm]{Definition} %真ん中の[thm]はthmと同じ通し番号を振るということ
    \newtheorem{lem}[thm]{Lemma}
    \newtheorem{rem}[thm]{Remark}
    \newtheorem*{rem*}{Remark} %通し番号を振らないときは*をつける
    \newtheorem{cor}[thm]{Corollary}
    \newtheorem{prop}[thm]{Proposition}
    \newtheorem{fact}[thm]{Fact}

    %定理番号は1.1から順番に自動で振られるが\setcounter{section}{4}などと入力することで途中からの番号をつけることができる


    \title{線形写像と行列の関係}
    \author{中井竜太郎\\\small{学籍番号 0530-37-0919}}     %これらはあとで\maketitleでタイトルに表示する

    %\pagestyle{headings}  %ページスタイルの設定(別に要らない)

    %-------------------------------------------------------------------------------------------------------
\begin{document}%ここから本文を書き始める。コマンド等を何も使わなければそのまま文字が出力される
    \begin{dfn}
        $f\in L^1(\Omega)$とする
       $$\int_{\Omega} |Df| \coloneqq \sup \left\{\int_{\Omega} f \mathrm{div} gdx \Big|g=(g_1,g_2,...,g_n)\in C_c^1(\Omega : \mathbb{R}^n ),|g(x)|\leqq1 ~ \text{for}~ x\in \Omega\right\}$$
       と定める$\int_{\Omega} |Df|< \infty $の時,$f$は有界変動であるという.\\
       $f\in L^1(\Omega)$で有界変動であるもの全体の集合を$BV(\Omega)$とかく.
    \end{dfn}
    \begin{prop}
    $f\in W_{loc}^{1,1}(\Omega)$のとき
    $$\int_{\Omega} |Df|=\int_{\Omega} |Df|dx$$
    ここで右辺の$Df$は弱微分の意味での$f$のgradientを指す.\\
    特に$f\in C^1(\Omega)$の時に上式が成り立つ.
    \end{prop}
    \begin{rem}
    a

    \end{rem}
    
$\mu$
$$\int_{\mathbb{R}^n} \phi(x) \cdot d\mu_{E_t}(x) = t^{1-n} \int_{\mathbb{R}^n} \phi(y/t) \cdot d\mu_E(y) \quad (*)$$ --- ### 1. ベクトル値Radon測度とその「作用」 まず、**ベクトル値Radon測度**(この場合は$\mu_{E_t}$や$\mu_E$)が数学的にどのように定義され、特定されるのかを理解することが重要です。 測度とは、直感的には集合に「大きさ」や「重み」(この場合はベクトル値)を与えるものですが、その厳密な性質は、**関数に対する「作用」によって完全に決まります**。これは、ある人物の性格が、様々な状況(インプット)に対してどう振る舞うか(アウトプット)で判断されるのに似ています。 ベクトル値Radon測度 $\mu$ の「作用」とは、任意の**テスト関数**(コンパクトな台を持つ連続なベクトル場 $\phi(x)$)を受け取って、スカラー値(実数)を返す操作、すなわち積分を指します。 $$\phi \mapsto \int_{\mathbb{R}^n} \phi(x) \cdot d\mathbf{\mu}(x)$$ 数学の基本的な定理(リースの表現定理)により、この「作用」の仕方さえ分かれば、測度 $\mathbf{\mu}$ は**一意に定まります**。 したがって、二つの測度 $\mathbf{\mu}_1$ と $\mathbf{\mu}_2$ が**等しい** ($\mathbf{\mu}_1 = \mathbf{\mu}_2$) とは、**ありとあらゆる**テスト関数 $\phi$ に対して、その作用の結果が全く同じになることを意味します。 $$\mathbf{\mu}_1 = \mathbf{\mu}_2 \iff \int_{\mathbb{R}^n} \phi \cdot d\mathbf{\mu}_1 = \int_{\mathbb{R}^n} \phi \cdot d\mathbf{\mu}_2 \quad (\text{for all } \phi \in C_c(\mathbb{R}^n; \mathbb{R}^n))$$ --- ### 2. 「変換された測度」の定義 次に、ご質問の箇所にある「測度 $\mu_E$ を変数変換 $y \mapsto y/t$ した上で係数 $t^{1-n}$ を掛けたもの」という、少し長い名前の測度を、仮に $\mathbf{\nu}$ と名付けましょう。 この新しい測度 $\mathbf{\nu}$ は、どのような測度なのでしょうか? その正体は、やはりテスト関数 $\phi$ に対する「作用」によって定義されます。その作用を、等式 $(*)$ の右辺の形をそのまま使って、次のように定義します。 $$\int_{\mathbb{R}^n} \phi(x) \cdot d\mathbf{\nu}(x) := t^{1-n} \int_{\mathbb{R}^n} \phi(y/t) \cdot d\mu_E(y)$$ この定義式の意味を見てみましょう。 * 測度 $\mathbf{\nu}$ にテスト関数 $\phi$ を作用させるとは、 * まず $\phi$ の変数を $x$ から $y/t$ に変換し(これが**変数変換**の部分)、 * それを元の測度 $\mu_E$ で積分し、 * 最後に定数係数 $t^{1-n}$ を掛ける操作である、と定めているのです。 --- ### 3. 等式の結論 これで準備が整いました。改めて証明で導かれた等式 $(*)$ を見てみましょう。 $$\underbrace{\int_{\mathbb{R}^n} \phi(x) \cdot d\mu_{E_t}(x)}_{\text{測度 } \mu_{E_t} \text{ の } \phi \text{ への作用}} = \underbrace{t^{1-n} \int_{\mathbb{R}^n} \phi(y/t) \cdot d\mu_E(y)}_{\text{測度 } \mathbf{\nu} \text{ の } \phi \text{ への作用}}$$ この等式は、左辺(測度 $\mu_{E_t}$ の作用)と右辺(我々が定義した測度 $\mathbf{\nu}$ の作用)が、**任意の**テスト関数 $\phi$ に対して常に等しいことを示しています。 これはまさに、セクション1で述べた「測度が等しいことの定義」そのものです。したがって、私たちは、これらの測度そのものが等しいと結論できます。 $$\mu_{E_t} = \mathbf{\nu}$$ この $\mathbf{\nu}$ を元の日本語の記述に戻すと、 「測度 $\mu_{E_t}$ は、測度 $\mu_E$ を変数変換 $y \mapsto y/t$ した上で係数 $t^{1-n}$ を掛けたものと等しい」 という結論が得られるわけです。$$







    \end{document}